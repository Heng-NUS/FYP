\section{Background and related works}
Disease control and prevention is vital for the whole society. Traditional surveillance method adopted by the Centers for Disease Control and Prevention (CDC) is, scrutinizing outpatient records from hospitals and virological test results from laboratories, which notices the disease after it actually occurred \cite{schmidt2012trending}. However, if there is no forecast and hospitals are ill-prepared for a rush of patients, the reception of in-time treatment will be affected \cite{elkin2017network}, other severe consequences can also be imagined. Therefore, a robust disease forecast system is needed.

To predict the outbreak of disease in advance, massive efforts have been put. \cite{chen2017reality} monitored the changes of Realistic Contact Networks (RCNs) to predict the dynamic movement of disease, \cite{chen2017disease} used machine learning with previous illness records to predict the future outbreak, while some researchers tried to build a prediction system based on data from social media. 

According to \cite{lee2013real}, social media contains information related to healthcare, individual health issue, symptom. \cite{ginsberg2009detecting} shows that spikes in flu queries and disease breakout coincide. However, since queries has little or no limitation and even don not need an account, they cannot be regarded as reliable data \cite{schmidt2012trending}. Other social media platforms such as Twitter and Facebook have proven their value for Big Data analyze. Twitter data has been found to be useful for public health applications \cite{denecke2009valuable}, including: (1) monitoring diseases, (2) public reaction, (3) outbreaks or emergencies, (4) prediction, (5) lifestyle, and (6) geolocation of disease surveillance \cite{andreu2015big}. In addition, social media is prompt. According to \cite{elkin2017network}, over 645 million active Twitter users collectively post an average of 58 million tweets (micro-blogs no more than 140 characters long) per day in 2017, and the number is still growing. A practical example is that researcher use Twitter predicted flu outbreaks 1–2 weeks ahead of CDC’s surveillance average \cite{signorini2011use}. \cite{elkin2017network} also showed Twitter data aligns with CDC’s outpatient records. All the information I have read so far proves that such data is valuable in healthcare surveillance.  

\section{Motivation}
Previous work relying on social media has successfully proposed some novel methods. \cite{sadilek2012predicting} and \cite{salathe2011assessing} utilize time-series analysis on single geography, \cite{elkin2017network} provided a generalized solution to identify how contagious diseases diffuse across geographies. However, time-series analyze can be inaccurate in this scenario (they rely merely on the statistics of social media instead of its content). For example, top search about a certain disease could result from a celebrity’s illness \cite{schmidt2012trending}. Similar experimental result showed in \cite{elkin2017network}. During festivals, the number of tweets decreased, and the prediction accuracy was affected. Obviously, such results are not robust enough to be applied in the real world. It therefore makes sense to find a new method to enhance the whole system. In this project, I will start from previous works, and use different techniques such as machine learning, heuristic function, NLP to extract textual implication containing in such data and create a more feasible solution.
