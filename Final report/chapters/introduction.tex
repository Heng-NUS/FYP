\section{Background and motivation}
Disease control and prevention is vital for the whole society. Traditional surveillance method adopted by the Centers for Disease Control and Prevention (CDC) is, scrutinizing outpatient records from hospitals and virological test results from laboratories, which notices the disease after it actually occurred \cite{schmidt2012trending}. This means that there is latency between the appearing of a health-related event and the official report of the event. It obviously that shortening the latency or predicting the event in advance could benefit medical system and therefore benefit the whole society. 
\\\\
As a service of instant messaging, social medias could captures the latest events happening in different places. 
According to \cite{lee2013real}, social medias contain information related to healthcare, individual health issue, symptoms. \cite{ginsberg2009detecting} shows that spikes in flu queries and disease breakout coincide. However, since queries has little or no limitation and even don not need an account, they cannot be regarded as reliable data \cite{schmidt2012trending}. Other social media platforms such as Twitter and Facebook have proven their value for Big Data analyze. Twitter data has been found to be useful for public health applications \cite{denecke2009valuable}, including: (1) monitoring diseases, (2) public reaction, (3) outbreaks or emergencies, (4) prediction, (5) lifestyle, and (6) geolocation of disease surveillance \cite{andreu2015big}. In addition, social media is prompt. According to \cite{elkin2017network}, over 645 million active Twitter users collectively post an average of 58 million tweets (micro-blogs no more than 140 characters long) per day in 2017, and the number is still growing. A practical example is that researcher use Twitter predicted flu outbreaks 1–2 weeks ahead of CDC’s surveillance average \cite{signorini2011use}. \cite{elkin2017network} also showed Twitter data aligns with CDC’s outpatient records. All of them suggest that social medias can be used for health-event analysis.\\

In this project, we mainly focus on shortening the latency rather than predicting event in advance. The main idea is automatically extracting the signals of healthcare events through social medias, the events could be either known or novel. A property of social medias is that hot topics are changing over time, therefore, by keeping analyzing data generated in time slices(such as in one day), the latest health events can be detected. To archive this, we adopted two steps: (1) filtering irrelevant data; (2) grouping filtered data. In detail, we analyzed Twitter data (separated in days) and integrated two models in section \ref{sec:uncovering}: a supervised binary-classification model used to screen out health-related events from massive metadata, and an unsupervised topic model used for find latent topics of them. Especially, we proposed a new topic model which takes advantages from both Biterm Topic Model (BTM) \cite{yan2013biterm} and word embeddings.

\section{Related works}
Our work is detecting the healthcare events on short-text social medias (eg. Twitter), it combines a supervised classification model and an unsupervised topic model to detect events that are either known or unknown. Before us, massive efforts have been put to similar goals. \cite{serban2019real} proposed a general framework for handling similar tasks. It introduces the common pipeline of data preprocessing, management, integration, model evaluation, etc. Our general framework design learns from it. In contrast to us, they mainly focused on the forecasting of known diseases and symptoms, hence their system is integrated with a single multi-classification model. \cite{paul2011you,paul2012model,paul2014discovering,tuarob2013discovering} combined the machine learning and topic model to extract the disease-related topics, they trained binary classifiers and used topic model to group documents. The topic models they used are variations of LDA, to mitigate the data sparsity problem on short text, their models are highly customized for certain topics, and hence cannot detect the unknown topics. To model short text, \cite{yan2013biterm} proposed BTM, it trains parameters based on biterms rather than documents. However, like most traditional topic models, it receives input expressed in bag-of-words, and suffers from low-frequency problem. Word embeddings contain the hidden relationships among words and therefore can help topic models mitigate the problem. \cite{dieng2019topic} proposed a novel method of how to bring word embeddings in topic model and how to train the parameters. Our model combines the BTM and word embeddings based on this method.